\documentclass[border=2mm]{standalone}

\usepackage{../assets/preamble}

\begin{document}
\begin{circuitikz}
    \draw (0,0) node[rground] (GND) {};
    \draw (0,4) to[battery1, l_=3V, -*, name=V] (GND);
    \draw (0,4)
    to[short,*-] ++(4,0)
    to[R=R, name=R1] ++(0,-2)
    to[push button, l=S1, *-, name=S1] ++(0,-2)
    to[short] (0,0);
    
    \draw (4,4)
    to[short, *-] ++(2,0)
    to[R=R, name=R2] ++(0,-2)
    to[push button, l=S2, *-] ++(0,-2)
    to[short,-*] (S1end);
    
    \draw (0,4) to[short] ++(0,0.15) node[vcc] {VCC};
    \draw (R1end) [-Latex, color=fhtw-blue] -- ++(-0.5,0) node[left, color=fhtw-blue] {D31};
    \draw (R2end) [-Latex, color=fhtw-blue] -- ++(-0.5,0) node[left, color=fhtw-blue] {D1};
    
    \draw (12,4) node[qfpchip, num pins=48] (U) {Puck.js};

    \draw (U.pin 48) to[short] ++(0,0) node[vcc] {VCC};
    \draw (U.pin 47) to[short] ++(0,0) node[rground, rotate=180] {};
    \draw (U.pin 31) [-Latex, color=fhtw-blue] -- ++(0.5,0) node[right, color=fhtw-blue] {D31};
    \draw (U.pin 1) [-Latex, color=fhtw-blue] -- ++(-0.5,0) node[left, color=fhtw-blue] {D1};
\end{circuitikz}
\end{document}
